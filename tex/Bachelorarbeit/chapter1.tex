%
% chapter1.tex
%

\chapter{Einführung}
\label{cha:introduction}

In modernen Rechnernetzen und Rechenzentren hat sich der Bereich der Lastverteilung zu einem der wichtigsten Teilbereiche entwickelt. Hintergrund ist, dass mit steigender Anzahl von Internetnutzenden die entsprechenden Hardwareressourcen nicht nur zur Verfügung gestellt werden müssen, sondern vielmehr eben auch die effiziente Nutzung von Interesse geworden ist.\cite{tanenbaum2007modern} Dabei gibt es verschiedene Ansätze und auch verschiedene Punkte, an denen Maßnahmen getroffen werden können, um eine breite Nutzung zu realisieren. Auch vor dem Hintergrund der globalen Erwärmung ist es sinnvoll, die Hardwarenutzung zu optimieren, da somit Energie eingespart und fossile Mittel geschont werden können. \cite{barroso2007case} \newline \newline
So wurden bereits relativ früh skalierbare Systeme entwickelt. Ein klassisches Beispiel ist hierbei der Einsatz von mehreren Prozessoren in Webservern der frühen 90er Jahre. Während dieser Zeit gab es technologische Vorstöße im Bereich des Internets für zivile End-anwender, wodurch dementsprechend die Nachfrage nach bereitstellender Rechentechnik stieg. Somit wurde indirekt die Zeit der Mehrkern-Prozessoren eingeläutet, obwohl diese erst Jahre später entwickelt wurden. Einen sehr ähnlichen Ansatz verfolgen die Entwickelnden moderner Lastverteiler. Dabei wird Wert auf eine horizontale Skalierung gesetzt. Es werden Ressourcen auf horizontaler Ebene nebeneinander gesetzt, um den Gesamtdurchsatz des Systems zu erhöhen. Der dem entgegenstehenden Ansatz wäre die vertikale Lastverteilung, bei der im Wesentlichen die Punktleistung eines Rechners erhöht wird. Auch die horizontale Skalierung findet breite Anwendung. Am wichtigsten hierbei ist das Hochfrequenz-Trading am Aktienmarkt sowie High-Performance-Computing im Allgemeinen. \newline \newline Allerdings hat sich mit der zeitlichen Weiterentwicklung nicht nur das Interesse an Pa-rallelisierbarkeit gesteigert, sondern ebenfalls ein Interesse an Hardwarearchitekturen, die im Gegensatz zum klassischen Prozessor nicht nur einen allgemeinen Zweck erfüllen, sondern für spezifische Anwendungsbereiche entwickelt werden. Dabei kann durch das an die Anwendung angepasste architektonische Layout dermaßen optimiert werden, dass es nicht nur zur Leistungssteigerung selbst kommt, sondern eben auch zu einer deutlich besseren Energieeffizienz. Zudem haben die letzten Vorstöße im Bereich des maschinellen Lernens dazu geführt, dass eine Menge der traditionellen Rechenzentrumsarchitekturen zwar für die modernen GPU-Cluster funktionieren, es aber eine Menge an Potenzial und Optimierungsspielraum gibt. \cite{jouppi2017datacenter} \newline \newline Diese Arbeit behandelt eben genau eine dieser modernen Technologien, die zwar mit einer Idee für einen Anwendungszweck ins Leben gerufen wurde, aber deren tatsächliche Anwendung noch immer hinterherhinkt.